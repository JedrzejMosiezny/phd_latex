% Chapter Template

\chapter{Introduction} % Main chapter title

\label{introduction} % Change X to a consecutive number; for referencing this chapter elsewhere, use \ref{ChapterX}

\lhead{\emph{Introduction}} % Change X to a consecutive number; this is for the header on each page - perhaps a shortened title

%----------------------------------------------------------------------------------------
%	SECTION 1
%----------------------------------------------------------------------------------------
\section{Introduction and motivation}

Pollution by sound conducted by aircraft becomes one of the major concerns while introducing new aircraft to the market. Airframe noise and engine noise are considered as disturbing to population inhabiting the areas near the aerodrome, especially during take-off and landing operations. The concerns regarding the influence of the aircraft noise on the surrounding environment lead to development of international regulations defining the limits of the Effective Perceived Noise Levels for aircraft of certain type or family, maximum take off weight and engine count, as well as methods for measuring the aircraft noise. Although relatively easy to conclude, measuring aircraft noise as described in the regulations in consideration is achievable only when the aircraft is capable of flying, which, obviously, is at the final end of the process of introducing the aircraft to the market.

In order to asses the noise of the aircraft a set of engineering methods allowing predicting the noise of the aircraft is conceived. Such methods include, but are not limited to, vibroacousics describing the effect of vibrating elements of devices on generation of sound, and  aeroacoustics describing the effect of airflow and flow with solid boundary interaction on generation of sound.

Aeroacoustics as such is a relatively new branch of physics combining the classical approach to propagation of sound with mathematical formulation of fluid flow known from fluid mechanics. It is said that modern approach to aeroacoustics originated with the works of Lighthill \citep{Light1} \citep{Light2} in mid 1950s. Combining the mathematical formulation of equations governing the fluid flow with equations describing wave propagation, sound power and sound intensity allowed to asses the sound levels of turbulent flow.

Along with improvements of the computational methods in fields of computational fluid dynamics, general numerical methods and availability of computational resources, Computational Aero Acoustics (described further in this work as computational aeroacoustics) was developed as an engineering tool for assessing sound generated by fluid flow without the necessity of manufacturing the device and performing expensive experiments.

This thesis opens with a high level overview of commercial air traffic situation based on data provided by EUROSTAT. The development and changes in the IFR movements in the European Union airspace are presented along with the forecast prepared by EUROSTAT and EUROCONTROL. This is followed by a high level analysis of current requirements regarding the noise emissions of a non-military aircraft. Following is a brief study of components of an aircraft responsible for generating sound pollution during various stages of flight. It is established that one of the main sources of noise generated by the aircraft is the fan and first stage of the low pressure compressor and therefore this component shall be investigated further.

The main focus of this thesis is generation of sound by a single blade of a low pressure compressor. It is assumed, that identifying sources of noise in the compressor flow will provide an insight for designing an efficient compressor blade with reduced noise emission and, therefore, allow for designing a quieter jet engine which directly translates to a less noisy aircraft. A NASA R67 1st stage compressor blade is the test subject for the thesis. The given compressor was chosen as the operating parameters range is similar to a modern day first stage fan of a twin spool, large bypass turbofan engine and large quantities of experimental and validation data. A comparison between different aeroacoustic methods is provided and a modification to a direct noise analysis method is suggested. The blade for the compressor is than analyzed using the proposed method and results are visualized and presented. The work is concluded and suggestions for designing a compressor blade with reduced generation of noise are formulated, along with suggestions for further work within researching the generation of compressor noise. 

%----------------------------------------------------------------------------------------
%	SECTION
%----------------------------------------------------------------------------------------
\section{Growth of air traffic movements in early 20th century}

Statistical records on commercial air movement between countries in Europe since year 2000 is available from EUROSTAT. Information regarding number of passengers, flights, registered aircraft and number of airports in the $1^{st}$ decade of $20^{th}$ century is extracted to describe the increase in the movement of commercial aircraft. Furthermore, a brief analysis of the long term forecasts (up to year 2035) provide some predictions on the future air traffic. Both statistics of the past movements and long term forecasts show an increase in the commercial air traffic above Europe. Combining the increase of number of flight operations from existing airports with rapid urban sprawl leads to increasing noise pollution in the urban areas.

%-----------------------------------
%	SUBSECTION
%-----------------------------------
\subsection{Years 2000 -- 2010}

First number to be shown is the total number of aircraft in the EU (fig. \ref{num_aircraft}). Trends for passenger aircraft and total number of aircraft show a slow increase of these numbers. Chain growth ratio of number of aircraft (fig. \ref{num_aircraft2}) shows a rapid increase in year 2006 and stabilization afterwards. At this point situation on European skies is rather comforting. But comparing these numbers with figure \ref{num_pass} may change the perspective.

\begin{figure}[h!]
\centering % bo \centering nie wstawia dodatkowego odstępu
\includegraphics[width=0.85\textwidth]{Pictures/num_aircraft.png}
\caption{Number of aircraft registered in the EU (EUROSTAT)}
\label{num_aircraft}
\end{figure}

\begin{figure}[h!]
\centering % bo \centering nie wstawia dodatkowego odstępu
\includegraphics[width=0.85\textwidth]{Pictures/num_aircraft2.png}
\caption{Growth rate of number of aircraft registered in the EU}
\label{num_aircraft2}
\end{figure}

As by figure \ref{num_pass}, number of passengers is increasing, at average 50 to 70 million passengers every year. The growth rates (fig. \ref{num_pass2}) show large annual fluctuation of passenger count. Slowdown in year 2008 and decrease of passenger number in year 2009 are the aftermath of Financial Crisis of 2007-08. The situation recovers in years 2010-11. With roughly the same number of passenger airplanes (2007-11) this leads to rapid increase of passenger flights.

Existing fleet is supposed to deliver people and freight 24 hours a day, with very short overhaul periods, even shorter times to refuel and reload. Grounded aircraft generates loss instead of profit. Increasing number of flights generates yet another problem. All airplanes need an airfield with proper infrastructure to perform flight operations, basic maintenance, serve passengers and so on.

\begin{figure}[h!]
\centering % bo \centering nie wstawia dodatkowego odstępu
\includegraphics[width=0.85\textwidth]{Pictures/num_pass.png}
\caption{Total number of passengers in the EU (EUROSTAT)}
\label{num_pass}
\end{figure}

\begin{figure}[h!]
\centering % bo \centering nie wstawia dodatkowego odstępu
\includegraphics[width=0.85\textwidth]{Pictures/num_pass2.png}
\caption{Growth rate. Total number of passengers in the EU}
\label{num_pass2}
\end{figure}

Surprisingly, the number of main airports is not following the trend set up by number of passengers.  As seen on figure \ref{num_aerodrome} number of main airports (serving more than 150 000 passengers a year) remains roughly the same, with very slight increase in research period. Change in number of main airports (fig. \ref{num_aerodrome2}) is caused by varying number of passengers, rather than closing and opening airports. Decrease of number of airports in year 2009 is the aftermath of the economy crisis. Total number of airports varies among the years. This number includes small and medium airports such as club airfields and regional airports. Total number of airports (including registered regional and club airports) was under more influence of crisis and changed more rapidly.

\begin{figure}[h!]
\centering % bo \centering nie wstawia dodatkowego odstępu
\includegraphics[width=0.85\textwidth]{Pictures/num_aerodrome.png}
\caption{Number of airports (EUROSTAT)}
\label{num_aerodrome}
\end{figure}

\begin{figure}[h!]
\centering % bo \centering nie wstawia dodatkowego odstępu
\includegraphics[width=0.85\textwidth]{Pictures/num_aerodrome2.png}
\caption{Growth rate. Number of airports (self study)}
\label{num_aerodrome2}
\end{figure}

%-----------------------------------
%	SUBSECTION
%-----------------------------------
\subsection{Years 2010 -- 2035}

Trends presented in previous section may not give the full perspective. Years 2001-2012 show a significant increase in air traffic in Europe. Expansion of the EU in year 2004, Financial Crisis of 2007-2008, Arab Spring trampling through the Middle East and Northern Africa all have their reflection in air movements on the continent. These factors make the predictions harder than simple extrapolation of data.

In order to show how air traffic will change, EUROCONTROL prepares medium and long term forecast for IFR movements. These forecasts are used mainly for planning purposes for airlines.

Long term forecast, published in June 2013 covers IFR flight movements for years 2013-2035. Because of the range, the forecast is more robust and is divided into four scenarios \citep{growth_2013}

\begin{description}
\item[Scenario A: Global Growth (Technological Growth):] Strong economic growth in an increasingly globalized World, with technology used successfully to mitigate the effects of sustainability challenges such as the environment or resources availability.
\item[Scenario B:] Not covered in \citep{growth_2013}.
\item[Scenario C: Regulated Growth:] Moderate economic growth, with regulation reconciling the environmental, social and economic demands to address the growing global sustainability concerns. This scenario has been constructed as the most likely of the four, most closely following the trends.
\item[Scenario C’: Happy Localism:] this scenario is introduced to investigate an alternative path for the future. With European economies being more and more fragile, increasing pressure on costs, stricter environmental constraints, air travel in Europe would adapt to new global environment but taking an inwards perspective. There would be less globalization, more trade inside EU (e.g. Turkey joining Europe is important in this scenario). Also, slow growth of leisure travel to outside Europe, however certainly more inside EU. More point-to-point traffic within Europe. It does not mean that Europe does not grow or does not adapt to new technologies and innovation but its main focus is local. Although this scenario is mostly based on scenario C (as its name indicates), it also inherits some aspects of other scenarios like higher fuel prices or low business aviation traffic of scenario D.
\item[Scenario D: Fragmenting World:] A World of increasing tensions between regions, with more security threats, higher fuel prices, reduced trade and transport integration and knock-on effects of weaker economies.
\end{description}

%\begin{figure}[h!]
%\centering % bo \centering nie wstawia dodatkowego odstępu
%\includegraphics[width=0.85\textwidth]{Pictures/ec1.png}
%\caption{Average Annual Growth in IFR movements per State, 2019 vs. 2012 \citep{eurocontrol}}
%\label{ec1}
%\end{figure}
%
%\begin{figure}[h!]
%\centering % bo \centering nie wstawia dodatkowego odstępu
%\includegraphics[width=0.85\textwidth]{Pictures/ec2.png}
%\caption{Number of additional movements per day for each state (2019 vs. 2012) \citep{eurocontrol}}
%\label{ec2}
%\end{figure}

Scenario C: Regulated growth is considered to be the most likely at point of publishing the report. This forecast predicts 14.4 million flights in Europe 2035, which is 1.5 times more than in 2012. That creates an average growth of 1.8\% per year. Forecast predicts that in 2025 traffic growth will decelerate due to predicted economic slowdown and reaching the capacity of airports.

As in medium term forecast, growth is not uniform across Europe. Due to lower starting point in calculations, more growth is expected in Eastern countries.

This however is not the full view on the situation. While growth will be faster in the East (figure \ref{scenarioC}), it is still mainly the big western countries that will need to deal with the greatest increase in the number of flights (figure \ref{totalflights}).

\begin{figure}[h!]
\centering % bo \centering nie wstawia dodatkowego odstępu
\includegraphics[width=0.85\textwidth]{Pictures/scenC.png}
\caption{Average annual growth (scenario C: Regulated Growth) \citep{eurocontrol}}
\label{scenarioC}
\end{figure}

\begin{figure}[h!]
\centering % bo \centering nie wstawia dodatkowego odstępu
\includegraphics[width=0.85\textwidth]{Pictures/total2035.png}
\caption{Total traffic in 2035 \citep{eurocontrol}}
\label{totalflights}
\end{figure}

Presented forecasts show, that air traffic in Europe will grow significantly in the next few years. With no actions taken, two paths are available. On one hand, running such traffic on existing fleet with airports (reaching maximum capacity) located near city centers will create an environment filled with constant aircraft noise. On a preventive side, noise emission regulations are being tightened nearly every year. And the only way to conform strict regulations is to use state of the art engines and airplanes, because in this case, silence is golden.

%----------------------------------------------------------------------------------------
%	SECTION
%----------------------------------------------------------------------------------------
\section{Some excerpts from airworthiness regulations}

Any aircraft intended to carry people, both crew and passengers must fulfill rigorous airworthiness requirements including maximum stresses on the airframe, safety and ecological impact induced while the aircraft is in use. Initially, the requirements were created by the country of origin of aircraft manufacturer, but since aerial operations become international and intercontinental, the airworthiness regulations evolved. Current airworthiness requirements are developed, introduced and maintained by international organizations such as ICAO. Excerpts from some top level requirements regarding generation of noise by aircraft at different stages of flight are presented in further sections.

%-----------------------------------
%	SUBSECTION
%-----------------------------------
\subsection{CAEP regulations}

ICAO's current environmental activities are largely undertaken through the Committee on Aviation Environmental Protection (CAEP), which was established by the Council in 1983, superseding the Committee on Aircraft Noise (CAN) and the Committee on Aircraft Engine Emissions (CAEE).

The current structure of the Committee includes three working groups and four support groups. The working groups deal with the technical and operational aspects of noise reduction and mitigation, with the aircraft noise and emissions issues linked to airports and operations and with the technical and operational aspects of aircraft emissions. One support group provides information on the economic costs and environmental benefits of the noise and emissions options considered by CAEP, one addresses models and databases issues, one deals specifically with the ICAO Carbon Calculator and the last one is aimed at scientific understanding of aviation environmental impacts.
 
About once a year, CAEP meets as a Steering Group to review and provide guidance on the progress of the activities of the working groups. So far, CAEP has held eight formal meetings: in 1986 (CAEP/1), 1991 (CAEP/2), 1995 (CAEP/3), 1998 (CAEP/4,) 2001 (CAEP/5), 2004 (CAEP/6), 2007 (CAEP/7) and 2010 (CAEP/8). Each formal CAEP meeting produces a report with specific recommendations for the consideration of the ICAO Council. 

The Council acts on recommendations from CAEP in the light of any comments received from the Air Navigation Commission and, if there are economic aspects, from the Air Transport Committee. In the case of recommendations to introduce or amend Standards and Recommended Practices, there are established procedures for consulting States, after which the final decision rests with the Council.


%-----------------------------------
%	SUBSECTION
%-----------------------------------
\subsection{ICAO Annex 16}
Historically the oldest and presumably the most important regulations are stated in ICAO Annex 16 – Environmental Protection, Volume 1. First issue of this document was released in year 1981. At time of writing this article the latest issue is 6th, released in year 2011. Document contains standards (not strict requirements), Recommended Practices and Guide of the noise certification of aircraft that are operated in international air navigation, in accordance with the classification set out in the individual chapters: Each chapter describes different noise measurement points (Table \ref{annex16_1}) and noise levels for specific aircraft types:

\begin{enumerate}[a)]
\item Annex 16 Chapter 2 describes requirements for subsonic, jet engine propelled air-craft certified before 6th November 1977. With exceptions;
\item Annex 16 Chapter 3 describes requirements for:
\begin{itemize}
\item[-] Subsonic, jet engine propelled aircraft certified between 6th November 1977 and 1st January 2006,
\item[-] Propeller driven aircraft (MTOW over 8618 kg) certified between 1st January 1985 and 1st January 2006;
\end{itemize}
\item Annex 16 Chapter 4 describes requirements for:
\begin{itemize}
\item[-] Subsonic, jet engine propelled aircraft certified after 1st January 2006;
\item[-] Propeller driven aircraft (MTOW over 8618 kg) certified after 1st January 2006.
\end{itemize}
\end{enumerate}

Separate Chapters contain information on light aircraft (Annex 16 Chapter 7) and helicopters (Annex 16 Chapters 8 \& 11) and will not be discussed.

Table \ref{annex16_1} and \ref{annex16_2} contain a brief summarize of maximum noise levels and their measurement points. Noise levels and measurement points are not rigid. Maximum noise levels are logarithmic dependent from Maximum Take-Off Weight (MTOW) of certified aircraft. Highest noise levels are for heavier aircraft, with MTOW above 385 000 kg. Annex 16 fully describes weather requirements, flight procedures and equipment setup for proper measurements. Noise levels are presented in EPNdB (Effective Perceived Noise dB). This unit is not measurable in a direct manner. EPNdB calculations are based on measurements of noise level (measurements of acoustical pressure), spectrum of noise level and corrected with sustainability factors and noise damping of air (also dependent on weather). Methods on how to establish a measurement point, calculate correction factors from weather, wind, inaccurate measurement point are described in Annex or in its Addenda. Data presented below is an excerpt from chapters 2, 3 and 4 from 6th edition of ICAO Annex 16 (Table \ref{annex16_1}).

Maximum noise levels taken from Annex 16 are below 108 EPNdB. In comparison: Heavy traffic generates around 85dB, pneumatic road drill - circa 100dB, live rock concert generates circa 110-115dB noise. Exposition to noise level higher than 110dB for over 15 minutes may result in hearing damage. Short term (less than 10 minutes) exposure to 120 results in hearing damage, 130 dB is considered as a threshold of pain, 150dB causes eardrum rapture, while 194dB is considered as theoretical limit for sound barrier at 1 atmosphere of pressure.

Noise levels appear to be high. But such levels occur only in the nearest vicinity of the airport. Concerning that nearly any main airport in Europe is surrounded by a large perimeter, nearest housing areas are subjected to noise levels that are safe, but may be considered as annoying. Also, many airports create their own noise requirements and do not allow air traffic operations of aircraft not conforming to such.

\begin{table}[]
\centering
\caption{Noise measurement points per ICAO Annex 16 \citep{annex16}}
\label{annex16_1}
\resizebox{\textwidth}{!}{%
\begin{tabular}{l|lll} \toprule
\multicolumn{1}{c}{\multirow{2}{*}{Chapter}} & \multicolumn{3}{l}{Noise measurement point} \\
\multicolumn{1}{c}{} & Name & Distance & Point \\ \midrule
\multirow{3}{*}{2} & Sideway & 650m & On line parallel to runway where measured noise is max \\
 & Fly-by & 6.5km & On extent of runway axis measured from start of take-off \\
 & Approach & 2000m & From runway threshold below approach path \\ \midrule
\multirow{4}{*}{3, 4} & Sideway (jet) & 450m & On line parallel to runway where measured noise is max \\
 & Sideway (prop) & 650m & Below take-off path for take-off power climb \\
 & Fly-by & 6.5km & On extent of runway axis measured from start of take-off \\
 & Approach & 2000m & From runway threshold below approach path \\ 
\bottomrule
\end{tabular}%
}
\end{table}

\begin{table}[]
\centering
\caption{Maximum noise levels per ICAO Annex 16 \citep{annex16}}
\label{annex16_2}
\begin{tabular}{l|lrr} \toprule
\multicolumn{1}{c}{Chapter} & \multicolumn{1}{c}{Point} & \multicolumn{1}{c}{Engine count} & \multicolumn{1}{c}{Maximum EPNdB} \\ \midrule
\multirow{3}{*}{2} & Sideway & N/A & 108-102 \\
 & Fly-by & N/A & 108-93 \\
 & Approach & N/A & 108-102 \\ \midrule
\multirow{5}{*}{3, 4} & Sideway & N/A & 103-94 \\
 & \multirow{3}{*}{Fly-by} & 1 or 2 & 101-89 \\
 &  & 3 & 104-89 \\
 &  & 4 & 106-89 \\
 & Approach & N/A & 105-98 \\
\bottomrule
\end{tabular}
\end{table}


%-----------------------------------
%	SUBSECTION
%-----------------------------------
\subsection{Local regulations}

One of the airports with most strict noise requirements is London Heathrow. It is the third busiest airport in the world, serving more than 70 million passengers in 2012 and handling more international passengers than any other airport in the world.

In order to prevent nearby housing areas from noise effects, particularly at night, Heathrow airport introduced their own regulations for incoming and departing traffic. Air traffic Control at Heathrow Approach Control guides traffic incoming from four major routes into one approach stream. When possible, Controllers advise the use of Continuous Descent Approach (CDA). CDA allows for a smooth, constant-angle descent to landing (Fig. \ref{cda}). A continuous descent approach starts ideally from the top of descent, i.e. at cruise altitude, and allows the aircraft flying its individual optimal vertical profile down to runway threshold.

\begin{figure}[h!]
\centering % bo \centering nie wstawia dodatkowego odstępu
\includegraphics[width=0.85\textwidth]{Pictures/cda.png}
\caption{CDA approach (solid line) vs. Standard approach (dash line)}
\label{cda}
\end{figure}

Night-time flights at Heathrow are subject to restrictions. Between 23:00 and 07:00, the noisiest aircraft (rated QC/8 and QC/16) cannot be scheduled for operation. In addition, during the night quota period (23:30–06:00) there are four limits:

\begin{itemize}
\item[-] A limit on the number of flights allowed;
\item[-] A quota count system which limits the total amount of noise permitted, but allows operators to choose to operate fewer noisy aircraft or a greater number of quieter planes;
\item[-] QC/4 aircraft cannot be scheduled for operation;
\item[-] A voluntary agreement with the airlines that no early morning arrivals will be scheduled to land before 04:30.
\end{itemize}

A trial of "noise relief zones" ran from December 2012 to March 2013, which concentrated approach flight paths into defined areas compared with the existing paths which were spread out. The zones used alternated weekly, meaning residents in the "no-fly" areas received respite from aircraft noise for set periods. However, it was concluded that some residents in other areas experienced a significant disbenefit as a result of the trial and that it should therefore not be taken forward in its current form.

The Quota Count (QC) system was introduced on Heathrow in 1993. Each aircraft is classified and awarded a grade, called a Quota Count, based on how much noise it generates. Quieter aircraft are given a smaller grade.  Aircraft are classified separately for landing and take-off. Take-off quota count values are based on the average of the certificated flyover and sideline noise levels at maximum take-off weight, with 1.75 EPNdB added for ICAO Annex 16 Chapter 2 aircraft. Landing quota count values are based on the certificated approach noise level at maximum landing weight minus 9.0 EPNdB.

Noise classification for aircraft is described in Table \ref{tab:qc1}. Examples of aircraft classified in the QC system are presented in Table \ref{tab:qc2}.

Noise levels required by Heathrow Airport are far stricter than those stated in ICAO Annex 16. Such restrictions result in relatively noise friendly environment around Heathrow. ERCD report 1101 – Noise Exposure Contours for Heathrow Airport \citep{ERCD}, prepared by Environmental Research and Consultancy Department of British Civil Aviation Authority shows the effect of Heathrow Airport traffic on nearby locations. Report prepared in year 2010, presents number and location of households affected by specific noise levels generated by Heathrow air traffic (Fig. \ref{heathrow}).

\begin{figure}[h!]
\centering % bo \centering nie wstawia dodatkowego odstępu
\includegraphics[width=0.85\textwidth]{Pictures/heathrow.png}
\caption{Heathrow noise contours. 83\% western \& 17\% eastern traffic \citep{ERCD}}
\label{heathrow}
\end{figure}

Noise contours (Fig. \ref{heathrow}) show that 83dB of perceived noise level contains the terminal area, 72dB -- in the nearest vicinity of the runway and 66dB outside the aerodrome premises. 60dB noise is comparable to normal office space or restaurant rustle. 70dB is the sound level of moderate traffic and may be annoying to some people.

\begin{table}[]
\centering
\caption{Noise Quota Count classification in Heathrow \citep{ERCD}}
\label{tab:qc1}
\begin{tabular}{@{}rl@{}}
\toprule
Noise Classification & Quota Count \\ \midrule
Below 84 EPNdB & Exempt \\
84-86.9 EPNdB & 0.25 \\
87-89.9 EPNdB & 0.5 \\
90-92.9 EPNdB & 1 \\
93-95.9 EPNdB & 2 \\
96-98.9 EPNdB & 4 \\
99-101.9 EPNdB & 8 \\
Greater than 101.9 EPNdB & 16 \\ \bottomrule
\end{tabular}
\end{table}

\begin{table}[]
\centering
\caption{Examples of QC aircraft classification \citep{ERCD}}
\label{tab:qc2}
\begin{tabular}{@{}rll@{}}
\toprule
Aircraft type & QA Departure & QC Arrival \\ \midrule
Airbus A320 family & 0.5-1 & 0.25-0.5 \\
Airbus A380 & 2 & 0.5 \\
Boeing 737 Classic & 0.25-0.5 & 1 \\
Boeing 747-400 & 4 & 2 \\
Boeing 747-8 & 2 & 1 \\
Boeing 757-200 & 0.5 & 0.25 \\
Boeing 767-300 & 1 - 2 & 1 \\
Boeing 777-200ER & 2 & 1 \\
Embraer 145 & 0.25 & 0.25 \\ \bottomrule
\end{tabular}
\end{table}

This concludes that Heathrow airport restricts the movements of really loud airships to prevent its neighbors from aircraft noise. Presumably, Heathrow restrictions will lower the acceptable noise levels even more, due to increasing air traffic and urban sprawl around the airport.

%----------------------------------------------------------------------------------------
%	SECTION
%----------------------------------------------------------------------------------------
\section{Aircraft propulsion noise generation}

Requirements described above are obviously relevant to and aircraft as a whole entity. In order to engineer a flying machine capable of fulfilling the stated requirements, one must identify the sources of noise within the aircraft in question.

Let's consider a "generic" airliner-kind of aircraft. Main sources of noise of such are the engines, aerodynamic surfaces and fuselage inducing turbulence and aerodynamically generated sound, APU as well as various internal aggregates (fuel and oil pumps, HVAC systems, electrical motors). 

Research provided by Traub \citep{traub} gives an insight towards the components of noise generated by the aircraft during takeoff and landing, that is during two flight stages where the sound induced by aircraft is the most audible by the population on the ground. During the take-off phase, with engines set to maximum thrust generation,  the fan noise and jet noise are the dominant components of the overall sound generation. During approach and landing phase, where engines are set to idle thrust, the dominant components are the engine fan and the airframe. The airframe noise is generated mostly by the extended high lift system generating flow of significant turbulence and vorticity \ref{traub1}.

The engine noise generation is distributed over it's main components \ref{traub1}. A rather obvious conclusion arises. The more accessible the component is from the outside, the higher noise levels of the component are registered. Following that statement, for a large bypass turbofan engine, the impact of fan and exhaust jet is the most significant during the take-off and landing phase.

\begin{figure}[h!]
\centering % bo \centering nie wstawia dodatkowego odstępu
\includegraphics[width=0.85\textwidth]{Pictures/traub1.png}
\caption{Sources of noise of a turbofan engine \citep{traub}}
\label{traub1}
\end{figure}

\begin{figure}[h!]
\centering % bo \centering nie wstawia dodatkowego odstępu
\includegraphics[width=0.85\textwidth]{Pictures/traub2.png}
\caption{Impact of aircraft noise sources at take-off and landing \citep{traub}}
\label{traub2}
\end{figure}

At this stage of study it can be assumed, that noise generation be the jet engine consists of three phenomena: noise generated by interaction of rotating machinery, combustion and combustion instability noise, and noise of mixing in shear layer in the exhaust jet. Detailed studies on sound generation by turbomachinery are found in set of articles in proceedings \cite{tonal}. Details on combustion noise are presented in book \citep{combustion}.

Reducing the noise of a jet engine is a daring engineering task. Yet, the easiest element to handle with regard to noise emission is to reduce the sound emission of exhaust jet by introducing the additional turbulence degrading the mixing in the shear layer of the air stream behind the engine. This is achieved by introducing a minor modification to an engine nacelle and outlet nozzle that enforces rapid dissipation of the shear layer between the ambient air and the engine exhaust jet (fig. \ref{chevron1}). 

\begin{figure}[h!]
\centering % bo \centering nie wstawia dodatkowego odstępu
\includegraphics[width=0.85\textwidth]{Pictures/chevron1.jpg}
\caption{Engine nacelle chevron on a GEnx engine}
\label{chevron1}
\end{figure}

Implementation of a chevron enforces fluid motion, so that with minor (below 0.5\%) decrease of thrust, achieving a decrease in EPNdB of a jet stream by 2.5 decibels, as performed on a scale tests performed by NASA (fig. \ref{chevron2}) \citep{methods}. Such devices are becoming more popular in modern (or retrofitted) airliners in order to conform the strict requirements.

\begin{figure}[h!]
\centering % bo \centering nie wstawia dodatkowego odstępu
\includegraphics[width=0.85\textwidth]{Pictures/chevron2.jpg}
\caption{Exemplary analysis on a chevron engine outlet \citep{methods}}
\label{chevron2}
\end{figure}

Reduction of noise generated by a fan requires deeper modifications to the engine construction. Both passive and active modifications to the fan rotor blades require extensive testing and certification on nearly all stages of the design process. 