% Chapter Template

\chapter{Conclusions \& Further work} % Main chapter title

\label{conclusions} % Change X to a consecutive number; for referencing this chapter elsewhere, use \ref{ChapterX}

\lhead{\emph{Conclusions \& Further work}} % Change X to a consecutive number; this is for the header on each page - perhaps a shortened title

%----------------------------------------------------------------------------------------
%	SECTION
%----------------------------------------------------------------------------------------
\section{Results discussion}
Initially, this thesis intended to investigate the following phenomena on a fan blade of a commercial large by-pass-ratio turbofan engine blade, yet no information on airfoil coordinates and efficiency figures of such devices are available for research purposes. It is assumed, that the phenomena that occur on such blade are similar in character to the ones obtained on the provided test case. Although the practical utilization of presented results is limited at this stage, some conclusion can be drawn out. The approach used on a NASA R67 transonic axial compressor provide some insight to noise generation phenomena in a "frozen-rotor" reference frame. It must be noted, that noise perceived by human or received by a microphone will be different from the ones presented. However, the noise perceived has it's source in the phenomena discussed in this thesis.

The main contributors to the blade's noise generation are the boundary layer separations induced by backflow in the boundary layer or by the shockwave boundary layer interaction. The existence of a shockwave in a device of that kind is also the main contributor of the so-called "buzzsaw noise", which may be recognized by simulating a rotor with a moving mesh and with a stationary receiver or data probe. Largest, in terms of amplitude and frequency, sound pressure fluctuations are found at the trailing edge of the blade and at trailing edge-tip junction. Sound pressure in decibel scale as well as average frequency per node of control surface are in expected range for this kind of device.

Following assumptions can be made. In order to reduce the noise generated by the transonic axial compressor blade one must eradicate the sources of separated flow and, if possible, provide a supersonic to subsonic transition of flow without inducing a shockwave. At time of preparation of this thesis, methods for inverse design of boundary shapes based on pressure and/or other design criteria exist in open source domain. Such software allows for generating airfoils for fixed wing aircraft and rotating machinery. Furthermore, modification of shape of the blade trailing edge by implementing a shape that reduces the vortex shedding or at least the turbulence intensity in wake. At this stage, there are no suggestions for manipulating the frequency of the fluctuations.

Results of direct noise analysis on hub and casing surfaces are not presented due to a simple mistake -- these surfaces were  accidentally removed from the CFD solution export of the DDES analysis. Only one low frequency cycle is captured due to another mistake. As described in chapter \ref{ddes}, the timestep of the analysis was decreased by the order of magnitude in order to improve the analysis convergence. However, the solution export was still set up to deliver a dataset per timestep which resulted in a dataset 10 times larger than necessary, with no available walltime to continue the calculation.

%----------------------------------------------------------------------------------------
%	SECTION
%----------------------------------------------------------------------------------------
\section{Improvements to the method}
Current approach may be compared to attempts of resolving a turbulent boundary layer with one finite volume element at the boundary wall. Although mathematically possible (and used in wall modeled turbulence models) it is valid under fulfilling some assumptions. For presented direct approach, it is assumed that sinusoidal fluctuation of wavelength enclosed by four finite volumes and resolved by four timesteps.

Resolving a sinusoidal fluctuation with four finite volumes and four timesteps fulfills the continuity and momentum equations, any change in temperature, resulting in changing velocity of sound will disrupt the assumptions presented in chapter \ref{approach}. FFT amplitude plots for ordinary frequencies, analyzed at random mesh points show, that the amplitudes of frequencies above the assumed by mesh and timestep range are dropped nearly to zero. A solution for this is using a finer mesh, capable of resolving shorter wavelength fluctuations, along with smaller timestepping to compensate for transition of sound wave in the finer mesh, especially in the boundary layer region and in the wake.

Considering that the mesh resolution is increased, a proper method for resolving turbulence flow is required. Although justified from th computational expense standpoint, the hybrid methods with shielding functions may result in filtering the source fluctuations at the boundary due to switching to the RANS part of given model. Using LES methods for further work will be considered.

Another aspect to be considered for improvement is the walltime management and data management. Presented analyses was performed on a Prometheus HPC and was distributed to 120 nodes. Normalised walltime for the analysis was above 1M CPU hours for both transition DDES and final DDES analyses and around 11TB of data was generated during the process. Using LES methods requires meshes of much greater density and therefore meshes of at least order of magnitude larger than used in this case. Assuming that walltime requirement is scaled linearly with the mesh size, assumed LES analysis of the presented test case, would have had required more than 10M CPU hours and around 1000CPU nodes.

Realtime processing of data to obtain sound information in the flowfield is impractical if even possible for full meshes, due to the amount of data generated. An efficient and high capacity storage storage is therefore required for performing direct approach analyses of sound on full flowfield. 

\section{Further work}
Sed ullamcorper quam eu nisl interdum at interdum enim egestas. Aliquam placerat justo sed lectus lobortis ut porta nisl porttitor. Vestibulum mi dolor, lacinia molestie gravida at, tempus vitae ligula. Donec eget quam sapien, in viverra eros. Donec pellentesque justo a massa fringilla non vestibulum metus vestibulum. Vestibulum in orci quis felis tempor lacinia. Vivamus ornare ultrices facilisis. Ut hendrerit volutpat vulputate. Morbi condimentum venenatis augue, id porta ipsum vulputate in. Curabitur luctus tempus justo. Vestibulum risus lectus, adipiscing nec condimentum quis, condimentum nec nisl. Aliquam dictum sagittis velit sed iaculis. Morbi tristique augue sit amet nulla pulvinar id facilisis ligula mollis. Nam elit libero, tincidunt ut aliquam at, molestie in quam. Aenean rhoncus vehicula hendrerit.

\section{Closing remarks}
Sed ullamcorper quam eu nisl interdum at interdum enim egestas. Aliquam placerat justo sed lectus lobortis ut porta nisl porttitor. Vestibulum mi dolor, lacinia molestie gravida at, tempus vitae ligula. Donec eget quam sapien, in viverra eros. Donec pellentesque justo a massa fringilla non vestibulum metus vestibulum. Vestibulum in orci quis felis tempor lacinia. Vivamus ornare ultrices facilisis. Ut hendrerit volutpat vulputate. Morbi condimentum venenatis augue, id porta ipsum vulputate in. Curabitur luctus tempus justo. Vestibulum risus lectus, adipiscing nec condimentum quis, condimentum nec nisl. Aliquam dictum sagittis velit sed iaculis. Morbi tristique augue sit amet nulla pulvinar id facilisis ligula mollis. Nam elit libero, tincidunt ut aliquam at, molestie in quam. Aenean rhoncus vehicula hendrerit.