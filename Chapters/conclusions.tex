% Chapter Template

\chapter{Conclusions \& Further work} % Main chapter title

\label{conclusions} % Change X to a consecutive number; for referencing this chapter elsewhere, use \ref{ChapterX}

\lhead{\emph{Conclusions \& Further work}} % Change X to a consecutive number; this is for the header on each page - perhaps a shortened title

%----------------------------------------------------------------------------------------
%	SECTION
%----------------------------------------------------------------------------------------
\section{Results discussion}
Initially, this thesis intended to investigate the following phenomena on a fan blade of a commercial large by-pass-ratio turbofan engine blade, yet no information on airfoil coordinates and efficiency figures of such devices are available for research purposes. It is assumed, that the phenomena that occur on such blade are similar in character to the ones obtained on the provided test case. Although the practical utilization of presented results is limited at this stage, some conclusion can be drawn out. The approach used on a NASA R67 transonic axial compressor provide some insight to noise generation phenomena in a "frozen-rotor" reference frame. It must be noted, that noise perceived by human or received by a microphone will be different from the ones presented. However, the noise perceived has it's source in the phenomena discussed in this thesis.

The main contributors to the blade's noise generation are the boundary layer separations induced by backflow in the boundary layer or by the shockwave boundary layer interaction. The existence of a shockwave in a device of that kind is also the main contributor of the so-called "buzzsaw noise", which may be recognized by simulating a rotor with a moving mesh and with a stationary receiver or data probe. Largest sound pressure fluctuations are found at the trailing edge of the blade and at trailing edge-tip junction.

Following assumptions can be made. In order to reduce the noise generated by the transonic axial compressor blade one must eradicate the sources of separated flow and, if possible, provide a supersonic to subsonic transition of flow without inducing a shockwave. Furthermore, modification of shape of the blade trailing edge by implementing a shape that reduces the vortex shedding or at least the turbulence intensity in wake.

%----------------------------------------------------------------------------------------
%	SECTION
%----------------------------------------------------------------------------------------
\section{Improvements to the method}
Sed ullamcorper quam eu nisl interdum at interdum enim egestas. Aliquam placerat justo sed lectus lobortis ut porta nisl porttitor. Vestibulum mi dolor, lacinia molestie gravida at, tempus vitae ligula. Donec eget quam sapien, in viverra eros. Donec pellentesque justo a massa fringilla non vestibulum metus vestibulum. Vestibulum in orci quis felis tempor lacinia. Vivamus ornare ultrices facilisis. Ut hendrerit volutpat vulputate. Morbi condimentum venenatis augue, id porta ipsum vulputate in. Curabitur luctus tempus justo. Vestibulum risus lectus, adipiscing nec condimentum quis, condimentum nec nisl. Aliquam dictum sagittis velit sed iaculis. Morbi tristique augue sit amet nulla pulvinar id facilisis ligula mollis. Nam elit libero, tincidunt ut aliquam at, molestie in quam. Aenean rhoncus vehicula hendrerit.

\section{Further work}
Sed ullamcorper quam eu nisl interdum at interdum enim egestas. Aliquam placerat justo sed lectus lobortis ut porta nisl porttitor. Vestibulum mi dolor, lacinia molestie gravida at, tempus vitae ligula. Donec eget quam sapien, in viverra eros. Donec pellentesque justo a massa fringilla non vestibulum metus vestibulum. Vestibulum in orci quis felis tempor lacinia. Vivamus ornare ultrices facilisis. Ut hendrerit volutpat vulputate. Morbi condimentum venenatis augue, id porta ipsum vulputate in. Curabitur luctus tempus justo. Vestibulum risus lectus, adipiscing nec condimentum quis, condimentum nec nisl. Aliquam dictum sagittis velit sed iaculis. Morbi tristique augue sit amet nulla pulvinar id facilisis ligula mollis. Nam elit libero, tincidunt ut aliquam at, molestie in quam. Aenean rhoncus vehicula hendrerit.