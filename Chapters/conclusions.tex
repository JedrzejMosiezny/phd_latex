% Chapter Template

\chapter{Conclusions \& Further work} % Main chapter title

\label{conclusions} % Change X to a consecutive number; for referencing this chapter elsewhere, use \ref{ChapterX}

\lhead{\emph{Conclusions \& Further work}} % Change X to a consecutive number; this is for the header on each page - perhaps a shortened title

%----------------------------------------------------------------------------------------
%	SECTION
%----------------------------------------------------------------------------------------
\section{Results discussion}
Initially, this thesis intended to investigate the sound generation phenomena on a fan blade of a commercial large by-pass-ratio turbofan engine blade, yet no information on airfoil coordinates and efficiency figures of such devices are available for research purposes. It is assumed, that the phenomena that occur on such blade are similar in character to the ones obtained on the provided test case. Although the practical utilization of presented results is limited at this stage, some conclusion can be drawn out. The approach used on a NASA R67 transonic axial compressor provide some insight to noise generation phenomena in a "frozen-rotor" reference frame. It must be noted, that noise perceived by human or received by a microphone will be different from the ones presented. However, the noise perceived has it's source in the phenomena discussed in this thesis.

The main contributors to the blade's noise generation are the boundary layer separations induced by backflow in the boundary layer or by the shockwave boundary layer interaction. The existence of a shockwave in a device of that kind is also the main contributor of the so-called "buzzsaw noise", which may be recognized by simulating a rotor with a moving mesh and with a stationary receiver or data probe. Largest, in terms of amplitude and frequency, sound pressure fluctuations are found at the trailing edge of the blade and at trailing edge-tip junction. Sound pressure in decibel scale as well as average frequency per node of control surface are in expected range for this kind of device.

Following assumptions can be made. In order to reduce the noise generated by the transonic axial compressor blade one must eradicate the sources of separated flow and, if possible, provide a supersonic to subsonic transition of flow without inducing a shockwave. At time of preparation of this thesis, methods for inverse design of boundary shapes based on pressure and/or other design criteria exist in open source domain. Such software allows for generating airfoils for fixed wing aircraft and rotating machinery. Furthermore, modification of shape of the blade trailing edge by implementing a shape that reduces the vortex shedding or at least the turbulence intensity in wake. At this stage, there are no suggestions for manipulating the frequency of the fluctuations.

Results of direct noise analysis on hub and casing surfaces are not presented due to a simple mistake -- these surfaces were  accidentally removed from the CFD solution export of the DDES analysis. Only one low frequency cycle is captured due to another mistake. As described in chapter \ref{cfd}, the timestep of the analysis was decreased by the order of magnitude in order to improve the analysis convergence. However, the solution export was still set up to deliver a dataset per timestep which resulted in a dataset 10 times larger than necessary, with no available walltime to continue the calculation.

The main downside of the presented study is lack of validation data for acoustical nearfield of the given case. It is assumed, that the approach is valid when two following factors coincide: if the CFD analysis that generate flow-field data is validated by experimental data and if FFT and RMS scripts return the expected results while the input is a sum of sine waves of known frequency, amplitudes and phase shifts. The CFD analysis results were validated with the experimental data provided in reference \citep{r67laser} as described in chapter \ref{cfd}, therefore it is assumed that the DDES flowfield data is valid. The RMS script return the expected average amplitude value while processing a set of sine wave signals, whereas the FFT script returned proper frequency spectrum and phase shift of the given signal. By this indirect validation, it is assumed that presented results are valid.

%----------------------------------------------------------------------------------------
%	SECTION
%----------------------------------------------------------------------------------------
\section{Improvements to the method}
Current approach may be compared to attempts of resolving a turbulent boundary layer with one finite volume element at the boundary wall. Although mathematically possible (and used in wall modeled turbulence models) it is valid under fulfilling some assumptions. For presented direct approach, it is assumed that sinusoidal fluctuation of wavelength enclosed by four finite volumes and resolved by four timesteps.

Resolving a sinusoidal fluctuation with four finite volumes and four timesteps fulfills the continuity and momentum equations, any change in temperature, resulting in changing velocity of sound will disrupt the assumptions presented in chapter \ref{approach}. FFT amplitude plots for ordinary frequencies, analyzed at random mesh points show, that the amplitudes of frequencies above the assumed by mesh and timestep range are dropped nearly to zero. A solution for this is using a finer mesh, capable of resolving shorter wavelength fluctuations, along with smaller timestepping to compensate for transition of sound wave in the finer mesh, especially in the boundary layer region and in the wake.

Considering that the mesh resolution is increased, a proper method for resolving turbulence flow is required. Although justified from the computational expense standpoint, the hybrid methods with shielding functions may result in filtering the source fluctuations at the boundary due to switching to the RANS part of given model. Using LES methods for further work will be considered.

Utilized method does not include the effects of aeroelasticity. The geometry and mesh is rigid and deformations due to aerodynamic loading are not computed. Deformation caused by the aerodynamic loading is variable and depends on the transient phenomena, such as flow separation, generation of Karmann vortices and pulsation of the shockwaves. These phenomena cause cyclic deformations to the compressor blade, which as a result cause the cyclic displacement of volume that is considered as acoustic source.

Another aspect to be considered for improvement is the walltime management and data management. Presented analyses was performed on a Prometheus HPC and was distributed to 120 nodes. Normalized walltime for the analysis was above 1M CPU hours for both transition DDES and final DDES analyses and around 11TB of data was generated during the process. Using LES methods requires meshes of much greater density and therefore meshes of at least order of magnitude larger than used in this case. Assuming that walltime requirement is scaled linearly with the mesh size, assumed LES analysis of the presented test case, would have had required more than 10M CPU hours and around 1000CPU nodes.

Realtime processing of data to obtain sound information in the flowfield is impractical if even possible for full meshes, due to the amount of data generated. An efficient and high capacity storage storage is therefore required for performing direct approach analyses of sound on full flowfield.

Post-processing of data obtained in this study was performed with use of Matplotlib Pyplot library available for Python scripting language and hence the information about mesh was dropped during the process. This is visible especially in the internal surface scatter plots where lack of interpolation between points is visible. Using a post-processing tool with custom filters such as Paraview, or implementing a mesh file reader into Python (or any other implementation) scripts will solve problem with interpolation of data and allow visualization of full flowfield datasets.

\section{Further work}
As stated above, lack of validation case that directly checks the acoustical nearfield results is somewhat of a challenge. Therefore developing an experimental and numerical case of known acoustic nearfield properties is required. 

The implementation of mesh reading or mapping to mesh functionality will be developed in further versions of the post-processing scripts. This will allow for more efficient qualitative an quantitative analysis of the obtained data.

Presented work describes a single passage of the compressor in stationary reference frame ("Frozen rotor") configuration and gives insight to basic aerodynamic phenomena contributing to noise. In order to asses the noise generation in compressor, or fan, flows the analysis ought to be performed with use of rotating mesh with sliding interfaces and include effects of rotor-to-stator interaction, as well as interaction with compressor hub and casing. Furthermore, performing data acquisition and processing on full flow-field of the given case will give a better opportunity to post-process the obtained data. By extending the analysis time to more than 1 cycle of low frequency sound (0.05s), and preferably to more than five of such cycles, it may be possible to obtain information on acoustical wave modes within the flowfield of the case subject to study.

Considering that the geometry of given case is periodic, assuming the periodicity of flow in the DDES and LES analysis is prone to over constraining the resulting flowfield. Hence the analyses should be conducted on full rotor-stator stage, or at least a large portion of a stage, to eradicate or minimize the effect of periodic boundary conditions. This, combined with proposed LES simulation, enforces using a mesh with cell count in $10^8$ order of magnitude, which leads to further challenges with data management and required computational resources.

Further improvement may contain including the aeroelastic phenomena in the analysis. As stated above, the phenomena related with cyclic deformation are expected to have a contribution to the noise flowfield. Stress calculations and transient mesh displacements are among the most challenging of numerical calculations. Considering the current requirements for the calculation, implementation of aeroelastic effects may cause a shortage in computational resources.

\section{Closing remarks}
The study presented in this thesis can be interpreted two-wise. One, as a case study on direct formulation of noise analysis and providing the minimum requirements towards the analysis. Second aspect is the study of generating sound by the transonic axial compressor rotor blade. Both aspects of the study are, at least partially, fulfilled.

The provided minimum requirements of the analysis proved to work at relatively high computational expense. Yet, using the acoustical analogies implemented into used CFD code with transonic flows with shockwaves is unjustified by the mathematical formulation of the FW-H analogy. The resolution of the obtained acoustical nearfield can be improved at a cost of higher computational power requirements.

The presented results of the compressors acoustical nearfield are rather expected and are relateable to relative mach number fields and time averaged static pressure fields. Results identify the sources of aerodynamic noise that are consistent with ones described in literature, turbulent regions of the flow are the source of pressure fluctuations of high amplitude and high frequency. Propagation of sound waves was captured for lower end of the frequency spectrum, while the effect of high pitch noise is captured as an increase in RMS sound pressure and decibel figures. These results can be used to design a compressor blade that generates lower aerodynamically induced noise.

The method will be improved in terms of computational efficiency and data storage and attempted to be used in other technical entities where noise generation and acoustical wave mode assessment is required.