% Chapter Template

\chapter{Current research on Computational Aeroacoustics} % Main chapter title

\label{CAA} % Change X to a consecutive number; for referencing this chapter elsewhere, use \ref{ChapterX}

\lhead{\emph{Current research on Computational Aeroacoustics}} % Change X to a consecutive number; this is for the header on each page - perhaps a shortened title

%----------------------------------------------------------------------------------------
%	SECTION
%----------------------------------------------------------------------------------------
\section{Classification of CAA methods}
Computational aeroacoustics is a branch of aeroacoustics that aims to analyze the generation of noise by turbulent flows by means of numerical methods. A following classification of available methods is currently in use.

\begin{enumerate}
   \item Hybrid Approach.
   \begin{enumerate}
     \item Integral Method
     \begin{enumerate}
     	\item Lighthill's Analogy
     	\item Kirchoff Integral
     	\item FW-H
     \end{enumerate}
     \item Linearized Euler Equations
     \item Pseudo Spectral
     \item EIF
     \item APE
   \end{enumerate}
   \item Direct approach.
\end{enumerate}

The direct approach is the core of this thesis and will be described in detail in chapter~\ref{approach}. A brief introduction to Lightill's analogy with Curle's modification is provided below. Ffowcs Williams Hawkings analogy as an extension to the theory is also provided.

%----------------------------------------------------------------------------------------
%	SECTION
%----------------------------------------------------------------------------------------
\section{Lighthill-Curle theory of aerodynamic sound}
A mathematically formulated linkage between description of fluid flow and sound generation phenomena was proposed and solved by M. J. Lighthill.  His work \citep{Light1} focused on sound generation as a byproduct of airflow as distinct from sound generated by vibration of solids.

Consider a system with fluctuating flow occupying a very large volume of fluid, at which the non fluctuating part is at rest. Three mechanism of introducing kinetic energy to the system and transforming it to "acoustic energy" are following:

\begin{itemize}
\item[I] By forcing a mass of the fluid in a fixed region to fluctuate, as in the loudspeaker diaphragm
\item[II] By forcing the momentum in fixed space to fluctuate or by forcing the rates of flux through a given control surface to vary, as in vibrating part of a machine (or after striking a tuning-fork)
\item[III] By forcing the rates of flux through a given control surface to vary, without the vibrating motion of solid boundaries, as in noise generated turbulence in flow.
\end{itemize}

Efficiency of transformation the kinetic energy do sound decreases down the list. First two phenomena are well established in current knowledge and were described in many sources. The research on sound generated aerodynamically starts (probably) with aforementioned work \citep{Light2}.

Lighthill proposes, that Reynolds momentum equation (derrived in chapter \ref{approach}) already expresses that the momentum changes at exactly the same rate as if the medium was at rest under the combined action of real stresses and fluctuating Reynolds stresses. Uniform acoustic medium at rest experiences stresses only from variation of density proportional to the speed of sound squared. A Lighthill stress tensor is therefore introduced to describe the fluctuations of the fluid medium subject to acoustic stresses:
\begin{equation} \label{eq:lighttensor}
T_{ij} = \rho v_i v_j + P_{ij} - a_0^2 \rho \delta_{ij}
\end{equation}

Term $P_{ij}$ is the compression tensor defined as:
\begin{equation} \label{eq:Ptensor}
P_{ij} = (p - p_0)\delta_{ij} - \sigma_{ij}
\end{equation}

\noindent where $\sigma_{ij}$ is the stress tensor due to molecular viscosity defined by:
\begin{equation} \label{eq:viscstress}
\sigma_{ij}
\equiv
\left[ \mu \left( \frac{\partial \overline{u_i}}{\partial x_{j}} + \frac{\partial \overline{u_j}}{\partial x_{i}} \right) \right]
- \frac{2}{3} \mu \frac{\partial \overline{u_l}}{\partial x_l} \delta_{ij}
\end{equation}

Propagation of sound in fluid medium without external forces is presented by following governing equations:

\begin{equation} \label{eq:caacont}
\frac{\partial \rho}{\partial t} + \frac{\partial}{\partial x_i} \left( \rho v_i \right) = 0
\end{equation}

\begin{equation} \label{eq:caacont2}
\frac{\partial}{\partial t} \left(\rho v_i \right) + a_0^2\frac{\partial \rho}{\partial x_i} = 0
\end{equation}

\begin{equation} \label{eq:caacont3}
\frac{\partial^2 \rho}{\partial t^2} - a_0^2 \nabla^2 \rho = 0
\end{equation}

The equation \ref{eq:caacont} is the continuity equation for a compressible fluid, equation \ref{eq:caacont2} is an approximate equation of momentum and equation \ref{eq:caacont3} is established by eliminating the $\rho v_i$ term from the previous equations.

%Consider the momentum equation for flow without external fluid forces:
%\begin{equation} \label{eq:caamom}
%\frac{\partial}{\partial t} \left(\rho v_i \right) + \frac{\partial}{\partial x_j}\left(\rho v_i v_j + P_{ij} \right)
%\end{equation}
%
By implementing the $T_{ij}$ tensor to the equations \ref{eq:caacont} thru \ref{eq:caacont3}, the following form is obtained: 

\begin{equation} \label{eq:caacont4}
\frac{\partial \rho}{\partial t} + \frac{\partial}{\partial x_i} \left( \rho v_i \right) = 0
\end{equation}

\begin{equation} \label{eq:caacont5}
\frac{\partial}{\partial t} \left(\rho v_i \right) + a_0^2\frac{\partial \rho}{\partial x_i} = - \frac{\partial T_{ij}}{\partial x_j}
\end{equation}

\begin{equation} \label{eq:caacont6}
\frac{\partial^2 \rho}{\partial t^2} - a_0^2 \nabla^2 \rho = \frac{\partial^2 T_{ij}}{\partial x_i \partial x_j}
\end{equation}

Therefore a term describing the fluctuations related to acoustic phenomena is now linked to the flow governing equations. It is now assumed, that resolving fluid flow along with the appropriate stress, strain and deformation terms can be now used to asses the sound phenomena in the flow field.

Should the receiver of the acoustical signal be outside the computational domain, further investigation must be concluded. Consider an unbounded flow field with a fluctuating point source, so that mass $Q(x, t)$ is introduced to the system at point $x$ and time~$t$, with total rate of introduction of $q(t)$. The density field is than given by the equation:

\begin{equation} \label{eq:densfield1}
\rho - \rho_0 = \frac{1}{4 \pi a_0^2} \frac{q' \left( \frac{t-r}{a_0} \right)}{r}
\end{equation}

\noindent where $r$ is the distance from source and $q'(t)$ is the time derivative of $q(t)$ and is defined as instantaneous source strength. For distributed source the equation \ref{eq:densfield1} takes form:

\begin{equation} \label{eq:densfield2}
\rho - \rho_0 = \frac{1}{4 \pi a_0^2} \int \frac{\partial}{\partial t} Q \left(y, t - \frac{\lvert x-y \rvert}{a_0} \right) \frac{dy}{\lvert x-y \rvert}
\end{equation}

The equation \ref{eq:densfield2} is then solved to a form:

\begin{equation} \label{eq:denssolved}
\rho - \rho_0 = \frac{1}{4 \pi a_0^2} \frac{\partial^2}{\partial x_i \partial x_j}
\int T_{ij} \left(y, t - \frac{\lvert x-y \rvert}{a_0} \right) \frac{dy}{\lvert x-y \rvert}
\end{equation}

The equation \ref{eq:denssolved} considers an unbounded flow field with point or volumetric source of fluctuations considered as quadrupole sources of acoustic fluctuations. This concept was evolved by Curle \citep{curle} to include the effect of solid boundaries on their reflection and diffraction. The modification of the original equation \ref{eq:denssolved} is:

\begin{equation} \label{eq:denscurle}
\begin{split}
\rho - \rho_0 = \frac{1}{4 \pi a_0^2} \frac{\partial^2}{\partial x_i \partial x_j}
\int_V T_{ij} \left(y, t - \frac{r}{a_0} \right) \frac{dy}{r} \\
- \frac{1}{4 \pi a_0^2} \frac{\partial}{\partial x_i}
\int_S P_i \left(y, t - \frac{r}{a_0} \right) \frac{dS(y)}{r}
\end{split}
\end{equation}

\noindent where:
\begin{equation} \label{eq:curlePi}
P_i = -l_j P_{ij}
\end{equation}

\noindent where: $l_i = (l_1, l_2, l_3) = n$ is the direction cosines of the outward normal from the fluid, and the sound generated in a medium at rest by a distribution of dipoles of strength $P_i$ per unit area and therefore $P_i$ is the force per unit area exerted on the fluid by the solid boundaries in the $x_i$ direction.


%----------------------------------------------------------------------------------------
%	SECTION
%----------------------------------------------------------------------------------------

\section{FW-H Analogy}
Further development of analogies developed in references \citep{Light1}, \citep{Light2} and \citep{curle} is presented in work \citep{FWH}. The extension to the theory includes the effect of arbitrary convective motion of fluid. More over, the FW-H analogy switches from Lighthill's unbounded fluid to a bounded volume. Thus it is possible to compute the flow phenomena within the acoustic near field (which in this case would be a CFD domain) and compute the sound propagation outwards to the acoustic far field (outside the CFD domain) using wave propagation equations.

FW-H analogy derives it's governing equation from a volume of fluid $V$ enclosed by a surface $\Sigma$, divided into regions 1 and 2 with surface of the discontinuity $S$ moving into region 2 with velocity $v$. By formulating the rate of change of mass within volume $V$ and deriving a generalized continuity and momentum equations, an equation governing the generation and propagation of sound is obtained.

\begin{equation} \label{eq:fwhgovern}
\begin{split}
\left( \frac{\partial^2}{\partial t^2} - a^2 \frac{\partial^2}{\partial x_i^2} \right)
\left( \overline{\rho - \rho_0} \right)
=
\frac{\partial^2 \overline{T_{ij}}}{\partial x_i \partial x_j}
- \frac{\partial}{\partial x_i} \left( P_{ij} \delta(f) \frac{\partial f}{\partial x_j} \right) \\
+ \frac{\partial}{\partial t} \left( \rho_0 v_i \delta(f) \frac{\partial f}{\partial x_i} \right)
\end{split}
\end{equation}

\noindent where: $\left( \overline{\rho - \rho_0} \right)$ is the generalized density perturbation - the amplitude of sound and $\overline{T_{ij}}$ is equal to $T_{ij}$ (eq. \ref{eq:lighttensor}) outside any surfaces and equal 0 when within them. Equation $f=0$ defines the division surface surface $S$, such that $f<0$ is in the region 1 and $f>0$ in region 2 (Heavyside function). The $\delta(f)$ is the Dirac delta function. 

The equation \ref{eq:fwhgovern} shows that sound can be regarded as generated by three source distributions: in volume - the quadrupole distribution of strength $T_{ij}$, on surface - the distribution of dipoles of strength density $P_{ij}n_j$ and monopole distributions from the displacement of volume by the moving surface.

Equation \ref{eq:fwhgovern} can be rewritten to a different form:

\begin{equation} \label{eq:fwhfluent}
\begin{split}
\frac{1}{a_0^2} \frac{\partial^2 \left(p'\right)}{\partial t^2}- \nabla^2  \left(p'\right)
= \frac{\partial^2}{\partial x_i \partial x_j} \lbrace T_{ij} H \left(f\right) \rbrace \\
- \frac{\partial}{\partial x_i} \lbrace \left[ P_{ij}n_j + \rho u_i \left( u_n - v_n \right) \right] \delta(f) \rbrace \\
+ \frac{\partial}{\partial t} \lbrace \left[ \rho_0 v_n + \rho \left( u_n - v_n \right) \right] \delta(f) \rbrace
\end{split}
\end{equation}

\noindent where:

\begin{description}
\item[$p' = p - p_0$] --- sound pressure fluctuation
\item[$u_i$] --- fluid velocity in the $x_i$ direction
\item[$u_n$] --- fluid velocity component normal to the surface $f = 0$
\item[$v_i$] --- surface velocity in the $x_i$ direction
\item[$v_n$] --- surface velocity component normal to the surface
\item[$H(f)$] --- Heaviside function
\item[$\delta(f)$] --- Dirac delta function
\end{description}

The rewritten equation \ref{eq:fwhfluent} represents an inhomogeneous wave equation can be integrated under specific assumptions and the solutions consists of surface (monopole and dipole sources) and volume integrals (quadrupole sources). Software package used for further computations ommits the effect of volume integral, therefore the result is of following form:

\begin{equation} \label{eq:fwhfluentsolved}
p'(x, t) = p'_T(x, t) + p'_L(x, t)
\end{equation}

\noindent with further development of the solution:

\begin{equation} \label{eq:fwhfluentpt}
\begin{split}
4 \pi p'_T (x, t)
= \int\displaylimits_{f=0} \left[ \frac{ \rho_0 \left( \dot{U_n} + U_{\dot{n}} \right)}{r \left( 1 - M_{r} \right)^{2}} \right] dS \\
+ \int\displaylimits_{f=0} \left[ \frac{\rho_0 U_n \lbrace r \dot{M_r} + a_0 \left( M_r - M^2 \right) \rbrace}{r^2 \left( 1 - M_{r} \right)^{3}} \right] dS
\end{split}
\end{equation}

\begin{equation} \label{eq:fwhfluentpl}
\begin{split}
4 \pi p'_L (x, t)
= \frac{1}{a_0} \int\displaylimits_{f=0} \left[ \frac{\dot{L_r}}{r \left(1 - M_r \right)^2 }\right] dS \\
+ \int\displaylimits_{f=0} \left[ \frac{L_r - L_M}{r^2 \left( 1 - M_r \right)^2 } \right] dS \\
+ \frac{1}{a_0} \int\displaylimits_{f=0} \left[ \frac{L_r \lbrace r \dot{M_r} + a_0 \left( M_r - M^2 \right) \rbrace }{r^2 \left( 1 - M_r \right)^3 } \right] dS
\end{split}
\end{equation}

\noindent where:

\begin{equation} \label{eq:fwhfluentui}
U_i = v_i + \frac{\rho}{\rho_0}\left(u_i - v_i \right)
\end{equation}

\begin{equation} \label{eq:fwhfluentli}
L_i = P_{ij} n_j + \rho v_i \left(u_i - v_i \right)
\end{equation}

When the integration surface coincides with an impenetrable wall, the two terms equation \ref{eq:fwhfluentsolved}, $ p'_T(x, t)$ and $p'_L(x, t)$ are often referred to as thickness and loading terms, respectively, in light of their physical meanings. The square brackets in equations \ref{eq:fwhfluentpt} and \ref{eq:fwhfluentpl} denote that the kernels of the integrals are computed at the corresponding retarded times, $\tau$, defined as in equation \ref{eq:fwhretard}, given the receiver time, $t$, and the distance to the receiver, $r$.

\begin{equation} \label{eq:fwhretard}
\tau = t - \frac{r}{a_0}
\end{equation}

The various subscripted quantities appearing in equations \ref{eq:fwhfluentpt} and \ref{eq:fwhfluentpl} are the inner products of a vector and a unit vector implied by the subscript. For instance, $L_r = \vec{L} \cdot \vec{r} = L_i r_i$ and $U_n = \vec{U} \cdot \vec{n} = U_i n_i$, where $\vec{r}$ and $\vec{n}$ denote the unit vectors in the radiation and wall-normal directions, respectively. The Mach number vector $M_i$ in equations \ref{eq:fwhfluentpt} and \ref{eq:fwhfluentpl} relates to the motion of the integration surface: $M_i = \dfrac{v_i}{a_0}$. The $L_i$ quantity is a scalar product $L_i M_i$. The dot over a variable denotes source-time differentiation of that variable \citep{fluenttheory} \citep{FWH} \citep{fwhaiaa}.

FW-H analogy is therefore the general form of Lighthill's acoustic analogy for aerodynamically generated noise, including volume sources of quadrupole kind, such as turbulence in free stream, and dipole and monopole sources of the moving solid body surface within the flow. Solution of the governing equation \ref{eq:fwhgovern} given in equations \ref{eq:fwhfluentsolved}, \ref{eq:fwhfluentpl}~\& \ref{eq:fwhfluentpt} omits the sources as weak. 

\section{Limitations to acoustic analogies}
Hybrid methods, including the presented Ffowks Williams -- Hawking analogy provide a computationally efficient task for engineering problems such as airframe noise, noise of jet injection to ambient medium at rest (that is jet engine noise problem), effect of wake generated by automobile mirror on noise in the vehicle cabin. The solution to the FW-H governing equation presented in equations \ref{eq:fwhfluentpt} and \ref{eq:fwhfluentpl} may reach instability when the Mach number in the sound radiation direction $M_r$ approaches 1, that is: when the freestream flow velocity approaches sonic conditions. Using the hybrid approach with acoustic analogies may be challenging and poses a risk of obtaining "non-physical" results for case considered in this thesis, that is blade of axial compressor in stationary reference frame.

For phenomena characteristic to axial compressor flow, that is strong shockwaves, shockwave with boundary layer interaction, high separation of flow enforced by shock waves, and high adverse pressure gradient may cause mathematical and numerical instabilities to the solution. Considering the pressure change within the computational domain, or, from the governing equation's standpoint, the volume enclosed by a surface, poses some difficulty to choosing free-stream values of density and pressure.

Further attempts towards gaining insight of the sound generation phenomena shall be performed by attempting to use a direct formulation method.